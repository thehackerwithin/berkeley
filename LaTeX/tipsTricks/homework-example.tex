% use the answers clause to get answers to print; otherwise leave it out.
%\documentclass[12pts, answers]{exam}
\documentclass[12pts]{exam}
\RequirePackage{amssymb, amsfonts, amsmath, latexsym, verbatim, xspace, setspace}

% By default LaTeX uses large margins.  This doesn't work well on exams; problems
% end up in the "middle" of the page, reducing the amount of space for students
% to work on them.
\usepackage[margin=1in]{geometry}
\usepackage{enumerate}
\usepackage{hyperref}

% Here's where you edit the Class, Exam, Date, etc.
\newcommand{\class}{NE 155}
\newcommand{\term}{Spring 2015}
\newcommand{\assignment}{HW 2}
\newcommand{\duedate}{2/11/14}
%\newcommand{\timelimit}{50 Minutes}

\newcommand{\nth}{n\ensuremath{^{\text{th}}} }
\newcommand{\ve}[1]{\ensuremath{\mathbf{#1}}}
\newcommand{\Macro}{\ensuremath{\Sigma}}
\newcommand{\vOmega}{\ensuremath{\hat{\Omega}}}

% For an exam, single spacing is most appropriate
\singlespacing
% \onehalfspacing
% \doublespacing

% For an exam, we generally want to turn off paragraph indentation
\parindent 0ex

%\unframedsolutions

\begin{document} 

% These commands set up the running header on the top of the exam pages
\pagestyle{head}
\firstpageheader{}{}{}
\runningheader{\class}{\assignment\ - Page \thepage\ of \numpages}{Due \duedate}
\runningheadrule

\begin{flushright}
\begin{tabular}{p{5in} r l}
NE 155 & Spring 2015 \\
Homework \#2 & Due February 11, 2014
\end{tabular}
\end{flushright}
\rule[1ex]{\textwidth}{.1pt}

%%%%%%%%%%%%%%%%%%%%%%%%%%%%%%%%%%%%%%%%%%%%%%%%%%%%%%%%%%%%%%%%%%%%%%%%%%%%%%%%%%%%%
%%%%%%%%%%%%%%%%%%%%%%%%%%%%%%%%%%%%%%%%%%%%%%%%%%%%%%%%%%%%%%%%%%%%%%%%%%%%%%%%%%%%%
On this homework:
\begin{itemize}
\item Show your work.
\item This homework should be done ``by hand" (i.e.\ not with a numerical program such as MATLAB, Python, or Wolfram Alpha) unless otherwise specified. You may use a numerical program to check your work. 
\item If using Python, be aware of \texttt{copy} vs.\ \texttt{deep copy}: \href{https://docs.python.org/2/library/copy.html}{https://docs.python.org/2/library/copy.html}
\item If you work with anyone else document what you worked on together.
\end{itemize}

% ---------------------------------------------
\begin{questions}
\addpoints
% intro
\question 
\begin{parts}
\part[1] Why are you taking this class? 
\ifprintanswers
  % no action
\else
  \vspace*{4 em}
\fi

\part[1] Is there a specific, non-grade related outcome you would like?
\ifprintanswers
  % no action
\else
  \vspace*{3 em}
\fi
\end{parts}

% ---------------------------------------------
\vspace*{2em}
\addpoints
% intro
\question 
\begin{parts}
\part[1] In your opinion, what was the most interesting development in the history of computing?
\ifprintanswers
  % no action
\else
  \vspace*{3 em}
\fi
  
\part[2] Why?
\ifprintanswers
  % no action
\else
  \vspace*{4 em}
\fi
\end{parts}


%------------------------------------
%------------------------------------
\newpage
\addpoints
% linear algebra review
\question[15] 
Determine which of the following matrices are non-singular and compute the inverse of these matrices:
% a values
\newcommand{\aaa}{4}
\newcommand{\aab}{1}
\newcommand{\aac}{6}
\newcommand{\aba}{3}
\newcommand{\abb}{0}
\newcommand{\abc}{7}
\newcommand{\aca}{-4}
\newcommand{\acb}{-1}
\newcommand{\acc}{-8}
% b values
\newcommand{\baa}{2}
\newcommand{\bab}{0}
\newcommand{\bac}{0}
\newcommand{\bba}{0}
\newcommand{\bbb}{-3}
\newcommand{\bbc}{0}
\newcommand{\bca}{0}
\newcommand{\bcb}{0}
\newcommand{\bcc}{\frac{1}{2}}
% c values
\newcommand{\caa}{1}
\newcommand{\cab}{1}
\newcommand{\cac}{-1}
\newcommand{\cad}{-1}
\newcommand{\cba}{1}
\newcommand{\cbb}{2}
\newcommand{\cbc}{-4}
\newcommand{\cbd}{-2}
\newcommand{\cca}{2}
\newcommand{\ccb}{1}
\newcommand{\ccc}{1}
\newcommand{\ccd}{5}
\newcommand{\cda}{-1}
\newcommand{\cdb}{0}
\newcommand{\cdc}{-2}
\newcommand{\cdd}{-4}
\begin{equation}
\text{a.} \begin{pmatrix}
   \aaa & \aab & \aac \\
   \aba & \abb & \abc \\
   \aca & \acb & \acc \\
\end{pmatrix} \qquad
%
\text{b.} \begin{pmatrix}
   \baa & \bab & \bac \\
   \bba & \bbb & \bbc \\
   \bca & \bcb & \bcc \\
\end{pmatrix} \qquad
%
\text{c.} \begin{pmatrix}
  \caa & \cab & \cac & \cad \\
  \cba & \cbb & \cbc & \cbd \\
  \cca & \ccb & \ccc & \ccd \\
  \cda & \cdb & \cdc & \cdd \\
\end{pmatrix} \nonumber
\end{equation}

%-------------------------------------------------------
%-------------------------------------------------------
\begin{solution}
A way to tell if $\ve{A}$ is singular that we discussed in class is that the determinant of $\ve{A}=0$.

\begin{parts}
\part To find the \ve{a} determinant, use the cofactor method and select $i$ or $j$ = 2 to include the zero entry.
%
\begin{align}
\det(\ve{a})_{j=2} = &\sum_{i=1}^3 a_{i2}(-1)^{i+2}M_{i2} \nonumber \\
%
= &a_{12} (-1)^3 M_{12} + a_{22} (-1)^4 M_{22} + a_{32} (-1)^5 M_{32} \nonumber \\
%
= &-\aab \cdot \det\begin{pmatrix}
   \aba & \abc \\
   \aca & \acc \\ \end{pmatrix} 
+ \abb \cdot \det\begin{pmatrix}
   \aaa & \aac \\
   \aca & \acc \\ \end{pmatrix} \nonumber \\
&- \acb \cdot \det\begin{pmatrix}
   \aaa & \aac \\
   \aba & \abc \\  \end{pmatrix}\nonumber \\
%
= &-\aab [(\aba \cdot \acc) - (\abc \cdot \aca)]% + \abb [(\aaa \cdot \acc) - (\aac \cdot \aca)] 
- \acb [(\aaa \cdot \abc) - (\aac \cdot \aba)] \nonumber \\
%
\det(\ve{a}) = & -1(-18 + 28) + 1(28 - 18) = \boxed{0 \rightarrow \text{singular}} \nonumber
\end{align}

%-------------------------------------------------------
\part Computing the determinant of \ve{b} is easy since it's diagonal:
\begin{align}
\det(\ve{b}) &= \baa ( \bbb \cdot \bcc ) - \bbb ( \baa \cdot \bcc ) + \bcc ( \baa \cdot \bbb ) \nonumber \\
\det(\ve{b}) &= -3 + 3 -3 = \boxed{-3 \rightarrow \text{non-singular}}\nonumber
\end{align}
%
Since \ve{b} is non-singular, we only need to compute it's inverse. Recall that for the inverse of a diagonal matrix $d^{-1}_{ii} = 1/d_{ii}$. Thus,
%
\begin{equation}
\ve{b}^{-1} = \boxed{\begin{pmatrix}
  \frac{1}{2} &  0 & 0 \\
  0 & -\frac{1}{3} & 0 \\
  0 &  0 & 2
\end{pmatrix}} \nonumber
\end{equation}


%-------------------------------------------------------
\part To find the \ve{c} determinant, use the cofactor method and select $i$ = 4 or $j$ = 2 to include the zero entry.
%
\begin{align}
\det(\ve{c})_{j=2} = &\sum_{i=1}^4 c_{i2}(-1)^{i+2}M_{i2} \nonumber \\
%
= &c_{12} (-1)^3 M_{12} + c_{22} (-1)^4 M_{22} + c_{32} (-1)^5 M_{32} + c_{42} (-1)^6 M_{42}\nonumber \\
%
= &-\cab \cdot \det\begin{pmatrix}
  \cba & \cbc & \cbd \\
  \cca & \ccc & \ccd \\
  \cda & \cdc & \cdd \\ \end{pmatrix} 
+ \cbb\cdot \det\begin{pmatrix}
  \caa & \cac & \cad \\
  \cca & \ccc & \ccd \\
  \cda & \cdc & \cdd \\ \end{pmatrix} \nonumber \\
&- \ccb \cdot \det\begin{pmatrix}
  \caa & \cac & \cad \\
  \cba & \cbc & \cbd \\
  \cda & \cdc & \cdd \\ \end{pmatrix}
+ \cdb \cdot \det\begin{pmatrix}
  \caa & \cac & \cad \\
  \cba & \cbc & \cbd \\
  \cca & \ccc & \ccd \\ \end{pmatrix} \nonumber \\
  %
\det(\ve{c}) = &\cdots = \boxed{0 \rightarrow \text{singular}} \nonumber
\end{align}

\end{parts}

\end{solution}



% --------------------------------------------
% ---------------------------------------------
% linear algebra review
\ifprintanswers
  % no action
\else
  \vspace*{2 em}
\fi
\addpoints
\question[15] Determine the eigenvalues and associated eigenvectors of the following matrices:
%
% a values
\renewcommand{\aaa}{3}
\renewcommand{\aab}{-1}
\renewcommand{\aba}{-1}
\renewcommand{\abb}{3}
% b values
\renewcommand{\baa}{2}
\renewcommand{\bab}{1}
\renewcommand{\bac}{0}
\renewcommand{\bba}{1}
\renewcommand{\bbb}{2}
\renewcommand{\bbc}{0}
\renewcommand{\bca}{0}
\renewcommand{\bcb}{0}
\renewcommand{\bcc}{3}
% c values
\renewcommand{\caa}{3}
\renewcommand{\cab}{2}
\renewcommand{\cac}{-1}
\renewcommand{\cba}{1}
\renewcommand{\cbb}{-2}
\renewcommand{\cbc}{3}
\renewcommand{\cca}{2}
\renewcommand{\ccb}{0}
\renewcommand{\ccc}{4}
\begin{equation}
\text{a.} \begin{pmatrix}
   \aaa & \aab \\
   \aba & \abb \\
\end{pmatrix} \qquad
%
\text{b.} \begin{pmatrix}
   \baa & \bab & \bac \\
   \bba & \bbb & \bbc \\
   \bca & \bcb & \bcc \\
\end{pmatrix} \qquad
%
\text{c.} \begin{pmatrix}
  \caa & \cab & \cac \\
  \cba & \cbb & \cbc \\
  \cca & \ccb & \ccc \\
\end{pmatrix} \nonumber
\end{equation}

% ---------------------------------------------
\begin{solution}
To find eigenvalues, we need to compute $\det(\ve{A} - \lambda \ve{I})=0$ for each system.

\begin{parts}
\part 
\begin{align}
\det(\ve{a} - \lambda \ve{I}) = &\det\begin{pmatrix}
   \aaa - \lambda & \aab \\
   \aba & \abb - \lambda \\
\end{pmatrix} = 0\nonumber \\
%
= &[(\aaa - \lambda)\cdot(\abb - \lambda) - (\aab \cdot \aba)] \nonumber \\
%
= & \lambda^2 -6 \lambda + 8\nonumber \\
%
= & (\lambda - 2)(\lambda - 4) = 0 \nonumber \\
%
&\boxed{\lambda = 2, 4} \nonumber
\end{align}
 
\newcommand{\lama}{2}
\newcommand{\lamb}{4}
 
And the associated right eigenvectors come from $(\ve{a} - \lambda_i \ve{I})\vec{x}_i = \vec{0}$. Use this formula to get the relationship among entries in the eigenvector set the entries. 

E.g., for $\lambda = \lama$:
%
\begin{align}
\begin{pmatrix}
   \aaa - \lama & \aab \\
   \aba & \abb - \lama \\
\end{pmatrix} 
\begin{pmatrix} x_1 \\ x_2 \end{pmatrix} &= 
\begin{pmatrix} 0 \\ 0 \end{pmatrix} \nonumber \\
%
 x_1 - x_2 &= 0  \nonumber \\
-x_1 + x_2 &= 0  \nonumber \\
%
\text{Thus } x_1 &= x_2\:, \qquad \text{Choose }x_1 = 1 \nonumber
\end{align}
This strategy then gives
\begin{equation}
\boxed{\vec{x}_{\lambda = \lama} = \begin{pmatrix} 1 \\ 1 \end{pmatrix} \qquad
\vec{x}_{\lambda = \lamb} = \begin{pmatrix} 1 \\ -1 \end{pmatrix}} \nonumber
\end{equation}

%------------------------------------
\part 
There are several ways to do this, but we'll again use the cofactors with $j=2$ (like question 1).
\begin{align}
\det(\ve{b} - \lambda \ve{I}) = &\det\begin{pmatrix}
   \baa - \lambda & \bab & \bac \\
   \bba & \bbb - \lambda & \bbc \\
   \bca & \bcb & \bcc - \lambda \\
\end{pmatrix}  = 0\nonumber \\
%
= &-\bab \cdot \det\begin{pmatrix}
   \bba & \bbc \\
   \bca & (\bcc - \lambda) \\ \end{pmatrix} 
+ (\bbb - \lambda) \cdot \det\begin{pmatrix}
   (\baa - \lambda) & \bac \\
   \bca & (\bcc - \lambda) \\ \end{pmatrix} \nonumber \\
&- \bcb \cdot \det\begin{pmatrix}
   (\baa - \lambda) & \bac \\
   \bba & \bbc \\  \end{pmatrix} \nonumber \\
%
= &-\bab [(\bba \cdot (\bcc - \lambda)) - (\bbc \cdot \bca)] + (\bbb - \lambda) [((\baa - \lambda) \cdot (\bcc - \lambda)) - (\bac \cdot \bca)] 
\nonumber \\
%&- \bcb [((\baa - \lambda) \cdot \bbc) - (\bac \cdot \bba)] \nonumber 
%
= & - \lambda^3 + 7 \lambda^2 - 15 \lambda + 9  \nonumber \\
%
= & (\lambda - 3)(\lambda - 3)(-\lambda + 1) = 0 \nonumber \\
%
&\boxed{\lambda = 3, 3, 1} \nonumber
\end{align}
%
%
To get the eigenvectors associated with the repeated eigenvalue we'll need to use free variables. 

First, row reduce $\ve{A} - 3 \ve{I})\vec{x} = \vec{0}$:
%
\begin{equation}
\begin{pmatrix}
   -1 &  1 & 0 \\
    1 & -1 & 0 \\
    0 &  0 & 0 \\
\end{pmatrix}  
\begin{pmatrix}
x_1 \\ x_2 \\ x_3
\end{pmatrix}  = 
\begin{pmatrix}
0 \\ 0 \\ 0
\end{pmatrix}
%
\rightarrow
%
\begin{pmatrix}
   -1 &  1 & 0 \\
    0 & 0 & 0 \\
    0 &  0 & 0 \\
\end{pmatrix}  
\begin{pmatrix}
x_1 \\ x_2 \\ x_3
\end{pmatrix}  = 
\begin{pmatrix}
0 \\ 0 \\ 0
\end{pmatrix} \nonumber
\end{equation}
%
Now we see that we will have two free variables (since we have 3 unknowns and 1 equation) and we need the relationship between the free variables and the fixed ones. Choose $x_2$ and $x_3$ to be free. We can see that
\[x_2 = x_1\]
and there is no relationship with $x_3$. 

Next, write a vector that contains this relationship and expand it into multiple eigenvectors - where each variable multiplies one of the vectors:
\[\begin{pmatrix}
x_2 \\ x_2 \\ x_3
\end{pmatrix} = 
x_2 \begin{pmatrix}
1 \\ 1 \\ 0
\end{pmatrix} + 
x_3 \begin{pmatrix}
0 \\ 0 \\ 1
\end{pmatrix}\]
% http://math.stackexchange.com/questions/144798/finding-eigenvectors-with-repeated-eigenvalues
There are our two eigenvectors for $\lambda = 3$. The other eigenvector is found as before.
%
\begin{equation}
\boxed{\vec{x}_{\lambda =3} = \begin{pmatrix} 0 \\ 0 \\ 1 \end{pmatrix} \qquad
\vec{x}_{\lambda =3} = \begin{pmatrix} 1 \\ 1 \\ 0 \end{pmatrix}\qquad
\vec{x}_{\lambda =1} = \begin{pmatrix} 1 \\ -1 \\ 0 \end{pmatrix}} \nonumber
\end{equation}

%------------------------------------
\part 
Just like part (b):
\begin{align}
\det(\ve{c} - \lambda \ve{I}) = &\det\begin{pmatrix}
   \caa - \lambda & \cab & \cac \\
   \cba & \cbb - \lambda & \cbc \\
   \cca & \ccb & \ccc - \lambda \\
\end{pmatrix}  = 0\nonumber \\
%
= &-\cab \cdot \det\begin{pmatrix}
   \cba & \cbc \\
   \cca & (\ccc - \lambda) \\ \end{pmatrix} 
+ (\cbb - \lambda) \cdot \det\begin{pmatrix}
   (\caa - \lambda) & \cac \\
   \cca & (\ccc - \lambda) \\ \end{pmatrix} \nonumber \\
&- \ccb \cdot \det\begin{pmatrix}
   (\caa - \lambda) & \cac \\
   \cba & \cbc \\  \end{pmatrix} \nonumber \\
%
= &-\cab [(\cba \cdot (\ccc - \lambda)) - (\cbc \cdot \cca)] + (\cbb - \lambda) [((\caa - \lambda) \cdot (\ccc - \lambda)) - (\cac \cdot \cca)] 
\nonumber \\
%&- \ccb [((\caa - \lambda) \cdot \cbc) - (\cac \cdot \cba)] \nonumber 
%
= & - \lambda^3 + 5 \lambda^2 + 2 \lambda -24  \nonumber \\
%
= & (\lambda - 4)(-\lambda + 3)(\lambda + 2) = 0 \nonumber \\
%
&\boxed{\lambda = 4, 3, -2} \nonumber
\end{align}
%
And the associated eigenvectors are
%
\begin{equation}
\boxed{\vec{x}_{\lambda =4} = \begin{pmatrix} 0 \\ 1 \\ 2 \end{pmatrix} \qquad
\vec{x}_{\lambda =3} = \begin{pmatrix} -1 \\ 1 \\ 2 \end{pmatrix}\qquad
\vec{x}_{\lambda =-2} = \begin{pmatrix} 3 \\ -8 \\ -1 \end{pmatrix}} \nonumber
\end{equation}

\end{parts}

\end{solution}


\end{questions}

\end{document}
